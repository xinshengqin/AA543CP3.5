\documentclass[11pt]{article}

\usepackage{graphicx}
\usepackage{amsmath,amsfonts,amssymb}

\usepackage{hyperref}  % for urls and hyperlinks
%\usepackage[pdftex]{graphicx} 
\usepackage{caption}
\usepackage{subcaption}
%\usepackage{enumitem}


\setlength{\textwidth}{6.2in}
\setlength{\oddsidemargin}{0.3in}
\setlength{\evensidemargin}{0in}
\setlength{\textheight}{8.7in}
\setlength{\voffset}{-.7in}
\setlength{\headsep}{26pt}
\setlength{\parindent}{0pt}
\setlength{\parskip}{5pt}


\title{Documentation to the code for computer project 3 in AA543 and some thoughts}

\author{Xinsheng Qin xsqin@uw.edu}

\begin{document}

\maketitle

\begin{abstract}
In this documentation, a detailed description to the code for computer project 3 in class AA543 is given.
This code includes modules for mesh generation, applying boundary condition and initial condition, and solver for solving some partial difference equations (PDE). 
At the end are some of my thoughts about this project.
\end{abstract}

\section{General Structure of the code}
This code is designed to solve for fluid fields, if given a domain (currently only 1D is available), initial condition on the domain, boundary condition on the boundary and the governing equations to be solved.
\par

A \textbf{\textit{mesh}} object should be first defined and initialized to specify the domain. 
Then initial condition is used to create a \textbf{\textit{velocityField}} object to specify a velocity field with initial value known. 
Finally, the \textbf{\textit{velocityField}} object is passed to one of the solver function defined in \textbf{\textit{solver\_1d.py}} to solve for fluid field until a final time. 

\section{Introduction to all modules in the code}
    \subsection{\textbf{\textit{mesh.py}}}
    
    \subsection{\textbf{\textit{velocityField.py}}}
    This file contains definition of \textbf{\textit{velocityField}} class.
        \subsubsection{Attributes}
        \begin{description}
            \item[$\bullet$] \textbf{mesh} : \textbf{mesh}
                \par
                A mesh object on which the velocity field is described.
            \item[$\bullet$] \textbf{u\_xmin} : \textbf{float}
                \par
                Velocity for left-most two ghost points at j=-1,0. For case where dirichlet boundary condition is chosen.
            \item[$\bullet$] \textbf{u\_xmax} : \textbf{float}
                \par
                Velocity for right-most two ghost points at j=J+1,J+2. For case where dirichlet boundary condition is chosen.
            \item[$\bullet$] \textbf{name} : \textbf{str}
                \par
                Name for this velocityField object.
            \item[$\bullet$] \textbf{u0} : \textbf{list}
                \par
                A list contains velocity values at each grid point.
            \item[$\bullet$] \textbf{u} : \textbf{numpy.array}
                \par
                A numpy array contains velocity values at each grid point and at each time step.
                The \textit{n-th} row contains velocity values at each grid point at time step $n-1$
                
        \end{description}

        \subsubsection{Methods}
%        \begin{description}[style=unboxed]
        \begin{description}
            \item[$\bullet$]\textbf{\_\_init\_\_}(mesh, IC, IC\_para1, IC\_para2, BC='periodic",
                \par
                u\_xmin=None, u\_xmax=None, name='default\_velocityField')
                \par
                Constructor for the class. 
            \item[$\bullet$] \textbf{applyBC}(%todo
                \par
        \end{description}
    \subsection{\textbf{\textit{solver\_1d.py}}}
\section{A Simple Sample to Demonstrate How to Implement the Code} 
\section{My Thoughts on This Project}

The code will be updated and new features will be added in. 
All latest code is public at GitHub: https://github.com/xinshengqin/AA543CP3.5.git. 

\end{document}
