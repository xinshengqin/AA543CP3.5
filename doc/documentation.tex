\documentclass[11pt]{article}

\usepackage{graphicx}
\usepackage{amsmath,amsfonts,amssymb}

\usepackage{hyperref}  % for urls and hyperlinks
%\usepackage[pdftex]{graphicx} 
\usepackage{caption}
\usepackage{subcaption}
%\usepackage{enumitem}


\setlength{\textwidth}{6.2in}
\setlength{\oddsidemargin}{0.3in}
\setlength{\evensidemargin}{0in}
\setlength{\textheight}{8.7in}
\setlength{\voffset}{-.7in}
\setlength{\headsep}{26pt}
\setlength{\parindent}{0pt}
\setlength{\parskip}{5pt}


\title{Documentation to the code for computer project 3 in AA543 and some thoughts}

\author{Xinsheng Qin xsqin@uw.edu}

\begin{document}

\maketitle

\begin{abstract}
In this documentation, a detailed description to the code for computer project 3 in class AA543 is given.
This code includes modules for mesh generation, applying boundary condition and initial condition, and solver for solving some partial difference equations (PDE). 
At the end are some of my thoughts about this project.
\end{abstract}

\section{General Structure of the code}
This code is designed to solve for fluid fields, if given a domain (currently only 1D is available), initial condition on the domain, boundary condition on the boundary and the governing equations to be solved.
\par

A \textbf{\textit{mesh}} object should be first defined and initialized to specify the domain. 
Then initial condition is used to create a \textbf{\textit{velocityField}} object to specify a velocity field with initial value known. 
Finally, the \textbf{\textit{velocityField}} object is passed to one of the solver function defined in \textbf{\textit{solver\_1d.py}} to solve for fluid field until a final time. 

\section{Introduction to all modules in the code}
    \subsection{\textbf{\textit{mesh.py}}}
    
    \subsection{\textbf{\textit{velocityField.py}}}
    This file contains definition of \textbf{\textit{velocityField}} class and several functions for setting initial condition for the class.
        \subsubsection{Attributes}
        \begin{description}
            \item[$\bullet$] \textbf{mesh} : \textbf{mesh}
                \par
                A mesh object on which the velocity field is described.
            \item[$\bullet$] \textbf{u\_xmin} : \textbf{float}
                \par
                Velocity for left-most two ghost points at j=-1,0. For case where dirichlet boundary condition is chosen.
            \item[$\bullet$] \textbf{u\_xmax} : \textbf{float}
                \par
                Velocity for right-most two ghost points at j=J+1,J+2. For case where dirichlet boundary condition is chosen.
            \item[$\bullet$] \textbf{name} : \textbf{str}
                \par
                Name for this velocityField object.
            \item[$\bullet$] \textbf{BC} : \textbf{str}
                \par
                Boundary condition for this velocity field.
            \item[$\bullet$] \textbf{u0} : \textbf{list}
                \par
                A list contains velocity values at each grid point.
            \item[$\bullet$] \textbf{u} : \textbf{numpy.array}
                \par
                A numpy array contains velocity values at each grid point and at each time step.
                The \textit{n-th} row contains velocity values at each grid point at time step $n-1$
                
        \end{description}

        \subsubsection{Methods}
%        \begin{description}[style=unboxed]
        \begin{description}
            \item[$\bullet$]\textbf{\_\_init\_\_}(mesh, IC, IC\_para1, IC\_para2, BC='periodic",
                \par
                u\_xmin=None, u\_xmax=None, name='default\_velocityField')
                \par
                Constructor for the class. 
            \item[$\bullet$] \textbf{applyBC}(i)
                \par
                This function assigns velocity values to 4 ghost points at time step $i$ according to value of the attribute \textbf{BC}. 
                Three types of boundary conditions are available currently: periodic B.C., dirichlet B.C., neumann B.C.
            \item[$\bullet$] \textbf{plot\_pointVsIndex}()
                \par
                Call matplotlib to plot the mesh. Save a png file plotting coordinates of each grid point versus its index in the list (index starts from 0) in current folder.
            \item[$\bullet$] \textbf{plot\_gapSpaceIndex}()
                \par
                Call matplotlib to plot the mesh. Save a png file plotting distance between each two points versus index of first point in each pair in the list (index starts from 0) in current folder.
            \item[$\bullet$] \textbf{plot\_uVsIndex}()
                \par
                Call matplotlib to plot the velocity field. Save a png file plotting velocity at each grid point versus index of the grid point in the list (index starts from 0) in current folder.
            \item[$\bullet$] \textbf{plot\_uVsX}(i,filename)
                \par
                Call matplotlib to plot the velocity field. 
                Save a png file with \textit{filename} as prefix of file name, plotting velocity at time step $i$ at each grid point versus coordinates of the grid point. 

        \end{description}
        \subsubsection{Other Functions}
        \begin{description}
            \item[$\bullet$]\textbf{IC\_gaussian}(xmin, xmax, xpeak, width, x)
                \par
                A function to be passed to the \textbf{\textit{velocityField}} class. 
                It will return velocity value at a grid point with coordinate $x$.
                \par
                \textit{xmin} is coordinate of the 1st grid point (j=1).
                \textit{xmax} is coordinate of the last grid point (j=J).
                \textit{xpeak} is coordinates of a grid point where peak of the gaussian wave should be located at.
                \textit{width} control width of the gaussian wave.
                \par
                The equation used to describe the gaussian wave is 
                $$ u(x) = 1 + exp({\frac{-10}{width}(\frac{(x-xpeak)*(xmax+xmin)}{xmax-xmin})^2}) $$

            \item[$\bullet$]\textbf{IC\_stepFunction}(xmin xmax, xpeak, width, x)
                \par
                %todo
            \item[$\bullet$]\textbf{IC\_sin}(xmin xmax, amplitude, circle\_frequency, x)
                \par
                A function to be passed to the \textbf{\textit{velocityField}} class. 
                It will return velocity value at a grid point with coordinate $x$.
                The equation used to describe velocity distribution is
                $$ u(x) = amplitude*sin(circle_frequency*x) $$
                Note: \textit{xmin} and \textit{xmax} are not used.

        \end{description}

    \subsection{\textbf{\textit{solver\_1d.py}}}
\section{A Simple Sample to Demonstrate How to Implement the Code} 
\section{My Thoughts on This Project}

The code will be updated and new features will be added in. 
All latest code is public at GitHub: https://github.com/xinshengqin/AA543CP3.5.git. 

\end{document}
